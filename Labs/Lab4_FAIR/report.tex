\documentclass{article}
    % General document formatting
    \usepackage[margin=0.7in]{geometry}
    \usepackage[parfill]{parskip}
    \usepackage[utf8]{inputenc}
    \usepackage{hyperref}
    % Related to math
    \usepackage{amsmath,amssymb,amsfonts,amsthm}

\begin{document}

Mélanie Fournier, 2024/09/15, Workflow and environment exercise.

\section*{Part 1}

For this exercise, I decided to revise the book/website I created for my Bachelor's thesis. 

I used quarto when writing my thesis, which allowed me to generate my thesis as a website/book as well as a PDF. The code was already uploaded on GitHub (\href{https://github.com/MelanieInky/ThesisBook/tree/master}{here}), and the associated website is hosted in a GitHub page as well. 

I also did my best to document the code I wrote as I went along writing it. 

What I updated was, in order:

\begin{itemize}
    \item Creating a new virtual environment linked to the project and downloading any dependencies
    \item The website had some broken reference, so I fixed it. I had to update some of my code because some part of it broke.
    \item I then generated the requirement.txt from the venv and updated my README with information about the version of quarto I used.    
\end{itemize}

These changes contribute to making sure my Bachelor thesis is as reproducible as possible. Using Quarto is also useful in the sense that most of the figures are generated by code, which participates in making the process more transparent. Furthermore, this allows me or anyone to quickly fix an issue with the code if there is one. There are a few figures where the code is not run, and I use a saved file instead. I don't quite remember why I did that, however.



\section*{Part 2 - FAIR data principles}


\subsection*{Findable}

The data is findable in the experiment data, the README explains how this data is generated. Each time data is generated, the parameters that were used to do so are available in another file in the same folder.

\subsection*{Accessible}

The data itself is accessible on the repo, as well as means to generate more. Permission is where I get stuck because while it is my work, it is also work done as a Bachelor's project, so I have no idea how it interacts with the university policy. I need to look more into it, so the data can be explicitly stated to be usable by other people.


\subsection*{Interoperable}

The data is available in a CSV file, which can easily be read by a computer. 

\subsection*{Reusable}

Using the code as explained in the README with the parameters as written in the accompanying metadata text file makes the experiment replicable. My project uses randomness and I unfortunately did not specify a seed to use. This is something that I would address from the get go in future projects.





\end{document}